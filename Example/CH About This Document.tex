    \chapter*{About This Document}
    \addcontentsline{toc}{chapter}{About This Document}

This document is a supplement the code comments.  It contains equations which are hard to express in code comments.  It also contains high level design intentions.

This document is for software developers is \emph{not} intended as a user's manual.

    \section*{Creation of This Document}
    \addcontentsline{toc}{section}{Creation of This Document}

This document was created using \LaTeX\index{Latex@\LaTeX}\@.  \LaTeX{} is standard with most \emph{Linux} installations. You can download a \emph{Windows} version as part of the \href{http://www.miktex.org/}{\emph{MiKTeX}} (www.miktex.org)\index{MiKTeX} project. A useful interface to \emph{MiKTeX} is provided by \href{http://www.winedt.com/}{\emph{WinEdt}}  (www.winedt.com). The \class{.dvi} viewer (\emph{YAP}) that comes with \emph{MiKTeX} requires the use of \href{http://www.cs.wisc.edu/~ghost/}{\emph{Ghostscript} and \emph{Ghostview}} (http://www.cs.wisc.edu/~ghost/) \index{Ghostscript}\index{Ghostview} in order to view \class{.dvi} file produced by \LaTeX\index{Latex@\LaTeX}~\cite{ref:kopka1999a}.

This document was built from many files.  The files or file types are listed below, along with a description.
    \begin{bulletedlist}
        \item The class file \class{hccbook.cls} which contains common formatting.
        \item A main \LaTeX{} file, \class{\documenttitle.tex}, that initializes the document and inputs the remaining source files.
        \item A bibliography file, \class{le.bib}, that contains references in a format \LaTeX{} understands.
        \item A bibliography formatting file, \class{hccplain.bst}.
        \item A \emph{WinEdt}\index{WinEdt} project file, \class{\documentshorttitle.prj} (not required to build the document), used by \emph{WinEdt} to store project settings.
        \item A batch file, \class{{\MakeUppercase makedocument}.bat}, used to build the file (but not required), see below for more information.
        \item \LaTeX{} chapter files, (\class{.tex} files prefixed with \class{CH}), included by the main file.
        \item \LaTeX{} section files, (\class{.tex} files prefixed with \class{SC}), included by the chapter files.
        \item Additional miscellaneous \LaTeX{} files used as needed, (\class{.tex, .sty}).  See comments in the source for explanations.
        \item Graphic files included by the \LaTeX{} files (see below for more information).
    \end{bulletedlist}

This file was built using the \class{\MakeUppercase makedocument.bat} batch file.  The batch file was created to allow passing additional parameters to \LaTeX{} and \emph{BibTeX}\@ that would otherwise require manually adding each time a build was performed.  (\emph{WinEdt} does not seem to provide this functionality.)  The parameter to include the directory for common files like \class{lebook.cls}, \class{le.bib}, and \class{leplain.bst} is passed this way.  These files were kept separate so that this manual and others can both access them (without keeping multiple copies).

    \subsection*{Additional Graphics Information}
    \addcontentsline{toc}{subsection}{Additional Graphics Information}

The \LaTeX{} file has been set up to produce both \class{.dvi} and \class{.pdf} formats as output.  There are a few issues that are associated with using graphics with both \class{.dvi} and \class{.pdf} files.

The first issue is that \LaTeX{} needs to know the size of graphics before it can include them.  With \class{.eps} files this is not an issue because the size (referred to as a ``bounding box'') is included in the graphic file.  However, to include other graphics files, the bounding box information is required.  A simple solution exists; generate the bounding box information using the batch file \class{bounding\ul{}box.bat}.  See the batch file itself for more details.

The \class{.dvi} format does not seem to handle \class{.jpg} files very well.  As a result all raster images are generated as both \class{.bmp} and \class{.jpg} formats.  The \class{.dvi} file uses the \class{.bmp} format and the \class{.pdf} files uses the \class{.jpg} format (to reduce file size).  Moreover, the \class{.pdf} files seem to require the \class{.jpg}s to be exactly 96 dpi so that they appear the correct size (without being stretched).  If needed, convert the \class{.jpg}s to 96 dpi using an image editor.  A free, versatile program \href{http://www.irfanview.com/}{IrfanView} works well (from the IrfanView menubar select Image, Resize/Resample to convert images).

The graphics used to generate the \class{.pdf} format must be in either \class{.pdf} or \class{.jpg} format.  Many of the figures used are of the \class{.eps} format (they are small, in vector format, and work well with \LaTeX) and as a result need to be converted to work in the \class{.pdf} output file.  Versions of the \class{.eps} files in \class{.pdf} format can be created the batch file \class{convert\ul{}eps\ul{}pdf.bat} located in the same directory as the figures. See the batch file itself for more details.

%    \section*{Nomenclature Used in This Manual}
%    \addcontentsline{toc}{section}{Nomenclature Used in This Manual}

\noindent Lance A. Endres\index{Endres, Lance A.} \\February 202x