\chapter*{About This Document}
\addcontentsline{toc}{chapter}{About This Document}


\section*{Creation of This Document}
\addcontentsline{toc}{section}{Creation of This Document}

This document was created using \LaTeX\index{Latex@\LaTeX}\@.  \LaTeX{} is standard with most \emph{Linux} installations. You can download a \emph{Windows} version as part of the \href{http://www.miktex.org/}{\emph{MiKTeX}} (www.miktex.org)\index{MiKTeX} project. A useful interface to \emph{MiKTeX} is provided by \href{http://www.winedt.com/}{\emph{WinEdt}}  (www.winedt.com). The \class{.dvi} viewer (\emph{YAP}) that comes with \emph{MiKTeX} requires the use of \href{http://www.cs.wisc.edu/~ghost/}{\emph{Ghostscript} and \emph{Ghostview}} (http://www.cs.wisc.edu/~ghost/) \index{Ghostscript}\index{Ghostview} in order to view \class{.dvi} file produced by \LaTeX\index{Latex@\LaTeX}~\cite{ref:kopka1999a}.

This document was built from many files.  The files or file types are listed below, along with a description.
\begin{bulletedlist}
	\item The class file \class{lebook.cls} which contains common formatting.
	\item A main \LaTeX{} file, \class{\documenttitle.tex}, that initializes the document and inputs the remaining source files.
	\item A bibliography files, \class{*.bib}, that contains references in a format \LaTeX{} understands.
	\item A \emph{WinEdt}\index{WinEdt} project file, \class{\documentshorttitle.prj} (not required to build the document), used by \emph{WinEdt} to store project settings.
    \item Batch files used to build the file.
	\item \LaTeX{} input files, (\class{.tex} files), included by the main file.
	\item Graphic files included by the \LaTeX{} files (see below for more information).
\end{bulletedlist}

\subsection*{Additional Graphics Information}
\addcontentsline{toc}{subsection}{Additional Graphics Information}

The \LaTeX{} file has been set up to produce both \class{.dvi} and \class{.pdf} formats as output.  There are a few issues that are associated with using graphics with both \class{.dvi} and \class{.pdf} files.

The graphics used to generate the \class{.pdf} format must be in either \class{.pdf} or \class{.jpg} format.  Some figures are of the \class{.eps} format (they are small, in vector format, and work well with \LaTeX) and as a result need to be converted to work in the \class{.pdf} output file.  Versions of the \class{.eps} files in \class{.pdf} format can be created the batch file
\begin{plainlist}
	\item \class{manual\ul{}convert\ul{}all\ul{}images.bat}
\end{plainlist}
located in the same directory as the figures.  See the batch file itself for more details.

\vspace{\baselineskip}%
\noindent Lance A. Endres\index{Endres, Lance A.} \\February 2023% 