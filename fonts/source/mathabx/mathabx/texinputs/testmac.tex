%
% TESTMAC.tex (december 20, 2003)
%
\catcode`\@=11
%
% Page setup
%
%\magnification=\magstep1
\voffset=0.12 true cm
\hoffset=0.31 true cm
\vsize=24.2 true cm
\hsize=15.3 true cm
\parindent=1.333em
%
% Heading
%
\font\nineit=cmti9
\font\twelverm=cmr12
\newif\iftitle
\titlefalse
\footline={\hfil}
\headline={\iftitle\hfil\global\titlefalse
	\ifodd\pageno\advance\pageno by 1\fi
	\else\vbox{\line{\ifodd\pageno
		{\nineit\@title}\hfil\llap{\tenrm\folio}
		\else\rlap{\tenrm\folio}\hfill{\nineit\@title}%
		\fi}\vskip6pt\hrule}\fi}
%
% Sectionning
%
\def\,{\ifmmode\mskip\thinmuskip\else\thinspace\fi}
\def\dots{\ifmmode\ldots\else%
	.\kern\fontdimen3\font
	.\kern\fontdimen3\font
	.\fi}
%
\def\french{\frenchspacing\language=1
	\def\@textsep{~---~}%
	\def\Author{Auteur~}\def\LastVersion{Derni\`ere version~}%
	\def\today{\day\space\ifcase\month\or
		janvier\or f\'evrier\or mars\or avril\or mai\or juin\or
		juillet\or ao\^ut\or septembre\or octobre\or novembre\or
		d\'ecembre\fi\space\number\day,\space\number\year}}%
%
\def\english{\frenchspacing\language=0
	\let\@textsep\enspace
	\def\Author{Author}\def\LastVersion{Last version}%
	\def\today{\ifcase\month\or
		January\or February\or March\or April\or May\or June\or
		July\or August\or September\or October\or November\or
		December\fi\space\number\day,\space\number\year}}%
%
\def\title#1{\gdef\@title{#1}}%
\def\subtitle#1{\def\@subtitle{#1}}%
\def\author#1{\def\@author{#1}}%
%
\def\maketitle{\global\titletrue{\parindent=0pt
	\leavevmode
	\ifx\@title\undefined\else{\twelverm\@title}\par\fi
	\vskip 6pt
	\hrule height 6pt
	\vskip 6pt
	\ifx\@subtitle\undefined\else\hfill{\twelverm\@subtitle}\par\fi
	\ifx\@author\undefined\else\hfill{\rm\Author: \@author}\par\fi
	\hfill{\rm\LastVersion: \today}}\vskip2\baselineskip}
%
\newcount\sectno
\newcount\subsectno
\def\section#1{\vskip\z@ plus\baselineskip
	\penalty-250\vskip\z@ plus-\baselineskip
	\bigbreak\noindent\ifx#1*%
	\def\@tmp##1{{\bf##1}\par\nobreak\medskip}\else
	\global\advance\sectno by 1
	\global\subsectno=0
	\def\@tmp{\the\sectno.\enspace{\bf#1}\par\nobreak\medskip}\fi
	\@tmp}%
\def\subsection#1{\medbreak\noindent\ifx#1*%
	\def\@tmp##1{{\bf##1\unskip.}\@textsep\ignorespaces}\else
	\global\advance\subsectno by 1
	\def\@tmp{\the\subsectno.\enspace{\bf#1\unskip.}%
	\@textsep\ignorespaces}\fi
	\@tmp}%
\def\description{\medbreak\bgroup
	\def\item##1{\medbreak\hangindent\parindent\leavevmode
		\hskip-\parindent{\bf##1.}\enspace\ignorespaces}%
	\def\subitem##1{\smallbreak\hangindent2\parindent\leavevmode
		{\it##1.}\enspace\ignorespaces}%
	\def\subsubitem##1{\par\hangindent3\parindent\leavevmode
		\hskip\parindent{##1.}\enspace\ignorespaces}%
	\let\itemitem=\subitem}%
\def\enddescription{\egroup\medbreak}%
\def\cs#1{\hbox{\tt\string#1}}%
%
% verbatim
%
\newdimen\tabindent\tabindent=2em
\let\verbatimsep\quad
{\obeyspaces\gdef {\leavevmode\space}%
\catcode`\^^I=\active\gdef^^I{\hskip\tabindent}}%
\def\rawverbatim{%
	\def\@numberedverbatimpar{\def\par{\ifvmode
		\endgraf\vskip\baselineskip\advance\count@ by 1
		\else\endgraf\fi}%
		\everypar={\advance\count@ by 1\leavevmode
		\llap{\the\count@\verbatimsep}}}%
	\def\@verbatimpar{\def\par{\ifvmode
		\endgraf\vskip\baselineskip\else\endgraf\fi}%
		\everypar={}}}%
\def\smartverbatim{%
	\def\@numberedverbatimpar{%
		\def\par{\ifvmode\endgraf
			\ifdim\lastskip<\medskipamount
			\removelastskip\penalty-100\medskip
			\@verbatimskiptrue\fi\else\endgraf\fi}%
		\everypar={\advance\count@ by 1\leavevmode
			\@ifverbatimskip
			\llap{\the\count@\verbatimsep}%
			\@verbatimskipfalse\fi}}%
	\def\@verbatimpar{%
		\def\par{\ifvmode\endgraf
			\ifdim\lastskip<\medskipamount
			\removelastskip\penalty-100\medskip\fi
			\else\endgraf\fi}%
		\everypar={}}}%
%
\def\@verbatimskiptrue{\let\@ifverbatimskip\iftrue}%
\def\@verbatimskipfalse{\let\@ifverbatimskip\iffalse}%
\smartverbatim
\def\verb{\bgroup\tt\uncatcodespecials\obeyspaces\@verb}%
\def\verbatim{\medbreak\bgroup
	\rightskip=0\p@ minus 1000\p@
	\tt\uncatcodespecials\obeyspaces
	\catcode`\^^I=\active\@verbatim}%
\def\@verbatim#1{\ifx#1[%
	\@verbatimskiptrue
	\@numberedverbatimpar\obeylines\let\@next=\@@verbatim
	\else\@verbatimpar\obeylines\parindent=0\p@
	\def\@next{\@@@verbatim#1}\fi\@next}%
\def\@@verbatim#1]{\setbox0=\hbox{#1\verbatimsep}\parindent=\wd0
	\count@=0\@@@verbatim}%
\def\verbatimfile#1{\ifx#1[\let\@next\@verbatimfile
	\else\def\@next{\@@verbatimfile{\input #1}}\fi\@next}%
\def\@verbatimfile#1]#2{\@@verbatimfile{[#1]\input #2}}%
{\catcode`\|=0\catcode`\\=12
|gdef|@verb#1\endverb{#1|egroup}%
|gdef|@@@verbatim#1\endverbatim{#1|egroup|medbreak}%
|gdef|@@verbatimfile#1{|verbatim #1|relax\endverbatim}}%
\def\uncatcodespecials{\def\do##1{\catcode`##1=12 }\dospecials}%
%
\long\def\comments#1\endcomments{}%
\def\fuzzytext{\tolerance=9999\hfuzz=3em\vfuzz=0.1pt}%
\def\normaltext{\tolerance=200\hfuzz=0.1pt\vfuzz=0.1pt}%
\def\newpage{\par\vfill\eject}%
\newdimen\shadeshift\shadeshift=1pt
\def\shadedtext#1{{\setbox0=\hbox{#1}\leavevmode
	\vtop to 0pt{\rlap{\special{color push rgb 0.75 0.75 0.75}%
	\kern0.1em\lower0.1em\copy0
	\special{color pop}}\vss}\box0}}%
\long\def\shadedparagraph#1\par{{\setbox0=\vbox{\hsize=\hsize#1}%
	\noindent\leavevmode
	\vtop to 0pt{\rlap{\special{color push rgb 0.75 0.75 0.75}%
	\kern0.1em\lower0.1em\copy0
	\special{color pop}}\vss}\box0\par}}%
%
% font tests
%
\newcount\n\newcount\m
\def\mixfrom#1to#2.{\medbreak\noindent
%	{\it (Random generator by Donald Arseneau.)}\par\noindent
	{\count255=#2%
	\advance\count255 by -#1%
	\n=0\loop\ifnum\n<25 \m=0%
	{\loop\ifnum\m<43\setrannum{\count@}{#1}{#2}%
		\char\count@
		\advance\m by 1
		\repeat}
	\advance\n by 1\endgraf\noindent
	\repeat}\medbreak}
%
\def\compare#1#2{\medbreak\noindent{%
	\font\ftestfont=#1\font\stestfont=#2%
	\count255=0%
	{\bf#1\unskip/\ignorespaces#2}
	\par\nobreak\medskip\noindent
	{\loop\ifnum\count255<256
	{\ftestfont\char\count255
	\stestfont\char\count255}
	\advance\count255 by 1\repeat}}\medbreak}
%
\def\docomparison#1#2from#3to#4.{\medbreak\noindent{%
	\font\ftestfont=#1\font\stestfont=#2%
	\count255=#3%
	\count0=#4\advance\count0 by 1
	{\bf#1\unskip/\ignorespaces#2}
	\par\medskip\noindent
	{\loop\ifnum\count255<\count0
	{\ftestfont\char\count255
	\stestfont\char\count255$_{_{\the\count255}}$}
	\advance\count255 by 1\repeat}}\medbreak}
%
\def\usuals#1{\medbreak\noindent{\bf#1\unskip.}~---~{%
	\font\testfont=#1\testfont\setbaselineskip
	\let\-=\allowbreak\unskip
	A\-B\-C\-D\-E\-F\-G\-H\-I\-J\-K\-L\-M\-%
	N\-O\-P\-Q\-R\-T\-S\-U\-V\-W\-X\-Y\-Z \-
	a\-b\-c\-d\-e\-f\-g\-h\-i\-j\-k\-l\-m\-%
	n\-o\-p\-q\-r\-s{}\-t\-u\-v\-w\-x\-y\-z \-
	\AE\-\OE\-\O
	\ae\-\oe\-\o\-\ss
	+\-=\-\# @ 1\-2\-3\-4\-5\-6\-7\-8\-9\-0 \-\$ \&  () []%
	\medbreak}}
\def\kerningtable[#1,#2][#3,#4]{\medbreak\noindent{\parindent=0pt
	\m=#4 \advance\m by -#3 \advance\m by 1
	\dim=\hsize \divide\dim by \m
	\m=#1 \advance\m by -1
	\n=#3 \advance\n by -1
	{\loop\leavevmode
		\ifnum\m<#2
		\advance\m by 1
		{\loop
			\ifnum\n<#4
			\advance\n by 1
			\setbox0\hbox{\char\m\char\n}%
			\setbox1\hbox{\char\m\null\char\n}%
			\hbox to \dim{\char\m\char\n
			\ifdim\wd0=\wd1\else{\rm*}\fi\hss}%
		\repeat}\endgraf
	\repeat}%
	}\medbreak}%
%
%%%%%%%%%%%%%%%%%%%%%%%%%%%%%%%%%%%%%%%%%%%%%%%%%%%%%%%%%%%%%%%%%%%%%%%

\hyphenation{prom-i-nent}

%\newcount\m \newcount\n
\newcount\p \newdimen\dim
\chardef\other=12

\def\hours{\n=\time \divide\n 60
  \m=-\n \multiply\m 60 \advance\m \time
  \twodigits\n\ heures \twodigits\m\ minutes}

\def\twodigits#1{\ifnum #1<10 0\fi \number#1}

\def\startfont{\tracinglostchars=0
	\fuzzytext
%	\tolerance=1000
%	\raggedbottom
	\parindent=0pt
%	\newlinechar=`@
%	\hyphenpenalty=200
%	\doublehyphendemerits=30000
	\font\testfont=\fontname
%	\spaceskip=0pt
	%  \leftline{{\bf\fontname\unskip}\ (tel que le \today, \`a \hours)}
	%  \nobreak\medskip\nobreak
	\testfont %\setbaselineskip
%	\ifdim\fontdimen6\testfont<10pt \rightskip=0pt plus 20pt
%	\else\rightskip=0pt plus 2em \fi
%	\spaceskip=\fontdimen2\testfont % space between words (\raggedright)
%	\xspaceskip=\fontdimen2\testfont
%	\advance\xspaceskip by\fontdimen7\testfont
	}

\def\setbaselineskip{\setbox0=\hbox{\n=0
	\loop\char\n \ifnum \n<255 \advance\n 1 \repeat}
	\baselineskip=6pt \advance\baselineskip\ht0
	\advance\baselineskip\dp0 }

\def\setchar#1{{\escapechar-1\message{\string#1 character = }%
	\def\do##1{\catcode`##1=\other}\dospecials
	\read-1 to\next
	\expandafter\finsetchar\next\next#1}}
\def\finsetchar#1#2\next#3{\global\chardef#3=`#1
	\ifnum #3=`\# \global\chardef#3=#2 \fi}
\def\promptthree{\setchar\background
	\setchar\starting \setchar\ending}

\def\mixture{\promptthree \domix\mixpattern}
\def\alternation{\promptthree \domix\altpattern}
\def\mixpattern{\0\1\0\0\1\1\0\0\0\1\1\1\0\1}
\def\altpattern{\0\1\0\1\0\1\0\1\0\1\0\1\0\1\0\1\0}
\def\domix#1{\par\chardef\0=\background \n=\starting
	\loop \chardef\1=\n #1\endgraf
	\ifnum \n<\ending \advance\n 1 \repeat}

\def\�{\discretionary{\background}{\background}{\background}}
\def\series{\promptthree \�\doseries\starting\ending\par}
\def\doseries#1#2{\n=#1\loop\char\n\�\ifnum\n<#2\advance\n 1 \repeat}
\def\complower{\�\doseries{`a}{`z}\doseries{'31}{'34}\par}
\def\compupper{\�\doseries{`A}{`Z}\doseries{'35}{'37}\par}
\def\compdigs{\�\doseries{`0}{`9}\par}
\def\alphabet{\setchar\background\complower}
\def\ALPHABET{\setchar\background\compupper}

\def\lowers{\docomprehensive\complower{`a}{`z}{'31}{'34}}
\def\uppers{\docomprehensive\compupper{`A}{`Z}{'35}{'37}}
\def\digits{\docomprehensive\compdigs{`0}{`4}{`5}{`9}}
\def\docomprehensive#1#2#3#4#5{\par\chardef\background=#2
	\loop{#1} \ifnum\background<#3\m=\background\advance\m 1
	\chardef\background=\m \repeat \chardef\background=#4
	\loop{#1} \ifnum\background<#5\m=\background\advance\m 1
	\chardef\background=\m \repeat}

\def\names{ {\AA}ngel\aa\ Beatrice Claire
	Diana \'Erica Fran\c{c}oise Ginette H\'el\`ene Iris
	Jackie K\=aren {\L}au\.ra Mar{\'\i}a N\H{a}ta{\l}{\u\i}e {\O}ctave
	Pauline Qu\^eneau Roxanne Sabine T\~a{\'\j}a Ur\v{s}ula
	Vivian Wendy Xanthippe Yv{\o}nne Z\"azilie\par}
\def\punct{\par\dopunct{min}\dopunct{pig}\dopunct{hid}
	\dopunct{HIE}\dopunct{TIP}\dopunct{fluff}
	\$1,234.56 + 7/8 = 9\% @ \#0\par}
\def\dopunct#1{#1,\ #1:\ #1;\ `#1'\ ?`#1?\ !`#1!\ (#1)\ [#1]\ #1*\ #1.\par}

\def\bigtest{\sample
	hamburgefonstiv HAMBURGEFONSTIV\par
	\names \punct \lowers \uppers \digits}

\def\math{\textfont1=\testfont \skewchar\testfont=\skewtrial
 \mathchardef\Gamma="100 \mathchardef\Delta="101
 \mathchardef\Theta="102 \mathchardef\Lambda="103 \mathchardef\Xi="104
 \mathchardef\Pi="105 \mathchardef\Sigma="106 \mathchardef\Upsilon="107
 \mathchardef\Phi="108 \mathchardef\Psi="109 \mathchardef\Omega="10A
 \def\ii{i} \def\jj{j}
 \def\\##1{|##1|+}\mathtrial
 \def\\##1{##1_2+}\mathtrial
 \def\\##1{##1^2+}\mathtrial
 \def\\##1{##1/2+}\mathtrial
 \def\\##1{2/##1+}\mathtrial
 \def\\##1{##1,{}+}\mathtrial
 \def\\##1{d##1+}\mathtrial
 \let\ii=\imath \let\jj=\jmath \def\\##1{\hat##1+}\mathtrial}
\newcount\skewtrial \skewtrial='177
\def\mathtrial{$\\A \\B \\C \\D \\E \\F \\G \\H \\I \\J \\K \\L \\M \\N \\O
 \\P \\Q \\R \\S \\T \\U \\V \\W \\X \\Y \\Z \\a \\b \\c \\d \\e \\f \\g
 \\h \\\ii \\\jj \\k \\l \\m \\n \\o \\p \\q \\r \\s \\t \\u \\v \\w \\x \\y
 \\z \\\alpha \\\beta \\\gamma \\\delta \\\epsilon \\\zeta \\\eta \\\theta
 \\\iota \\\kappa \\\lambda \\\mu \\\nu \\\xi \\\pi \\\rho \\\sigma \\\tau
 \\\upsilon \\\phi \\\chi \\\psi \\\omega \\\vartheta \\\varpi \\\varphi
 \\\Gamma \\\Delta \\\Theta \\\Lambda \\\Xi \\\Pi \\\Sigma \\\Upsilon
 \\\Phi \\\Psi \\\Omega \\\partial \\\ell \\\wp$\par}
\def\mathsy{\begingroup\skewtrial='060 % for math symbol font tests
 \def\mathtrial{$\\A \\B \\C \\D \\E \\F \\G \\H \\I \\J \\K \\L
	\\M \\N \\O \\P \\Q \\R \\S \\T \\U \\V \\W \\X \\Y \\Z$\par}
 \math\endgroup}

\def\oct#1{\hbox{\rm\'{}\kern-.2em\it#1\/\kern.05em}} % octal constant
\def\hex#1{\hbox{\rm\H{}\tt#1}} % hexadecimal constant
\def\setdigs#1"#2{\gdef\h{#2}% \h=hex prefix; \0\1=corresponding octal
 \m=\n \divide\m by 64 \xdef\0{\the\m}%
 \multiply\m by-64 \advance\m by\n \divide\m by 8 \xdef\1{\the\m}}
\def\testrow{\setbox0=\hbox{\penalty 1\def\\{\char"\h}%
 \\0\\1\\2\\3\\4\\5\\6\\7\\8\\9\\A\\B\\C\\D\\E\\F%
 \global\p=\lastpenalty}} % \p=1 if none of the characters exist
\def\oddline{\cr
	\noalign{\nointerlineskip}
	\multispan{19}\hrulefill&
	\setbox0=\hbox{\lower 2.3pt\hbox{\hex{\h x}}}\smash{\box0}\cr
	\noalign{\nointerlineskip}}
\newif\ifskipping
\def\evenline{\loop\skippingfalse
	\ifnum\n<256 \m=\n \divide\m 16 \chardef\next=\m
	\expandafter\setdigs\meaning\next \testrow
	\ifnum\p=1 \skippingtrue \fi\fi
	\ifskipping \global\advance\n 16 \repeat
	\ifnum\n=256 \let\next=\endchart\else\let\next=\morechart\fi
	\next}
\def\morechart{\cr\noalign{\hrule\penalty5000}
	\chartline \oddline \m=\1 \advance\m 1 \xdef\1{\the\m}
	\chartline \evenline}
\def\chartline{&\oct{\0\1x}&&\:&&\:&&\:&&\:&&\:&&\:&&\:&&\:&&}
\def\chartstrut{\lower4.5pt\vbox to14pt{}}
\def\table{\bigbreak\global\n=0
	\halign to\hsize\bgroup
		\chartstrut##\tabskip0pt plus10pt&
		&\hfil##\hfil&\vrule##\cr
		\lower6.5pt\null
		&\hbox to 0pt{\hss\rm\fontname\hss}
		&&\oct0&&\oct1&&\oct2&&\oct3&&\oct4
		&&\oct5&&\oct6&&\oct7&\evenline}
\def\endchart{\cr\noalign{\hrule}
	\raise11.5pt\null&&&\hex 8&&\hex 9&&\hex A&&\hex B&
	&\hex C&&\hex D&&\hex E&&\hex F&\cr\egroup\bigbreak\par}
\def\:{\setbox0=\hbox{\char\n}%
	\ifdim\ht0>7.5pt\reposition
	\else\ifdim\dp0>2.5pt\reposition\fi\fi
	\box0\global\advance\n 1 }
\def\reposition{\setbox0=\vbox{\kern2pt\box0}\dim=\dp0
	\advance\dim 2pt \dp0=\dim}
\def\centerlargechars{
	\def\reposition{\setbox0=\hbox{$\vcenter{\kern2pt\box0\kern2pt}$}}}

\def\text{{%\advance\baselineskip-4pt
%\setbox0=\hbox{abcdefghijklmnopqrstuvwxyz}
%\ifdim\hsize>2\wd0 \ifdim 15pc>2\wd0 \hsize=15pc \else \hsize=2.5\wd0 \fi\fi
On November 14, 1885, Senator \& Mrs.~Leland Stanford called
together at their San Francisco mansion the 24~prominent men who had
been chosen as the first trustees of The Leland Stanford Junior University.
They handed to the board the Founding Grant of the University, which they
had executed three days before. This document---with various amendments,
legislative acts, and court decrees---remains as the University's charter.
In bold, sweeping language it stipulates that the objectives of the University
are ``to qualify students for personal success and direct usefulness in life;
and to promote the publick welfare by exercising an influence in behalf of
humanity and civilization, teaching the blessings of liberty regulated by
law, and inculcating love and reverence for the great principles of
government as derived from the inalienable rights of man to life, liberty,
and the pursuit of happiness.'' \moretext
(!`THE DAZED BROWN FOX QUICKLY GAVE 12345--67890 JUMPS!)\par}}
\def\moretext{?`But aren't Kafka's Schlo{\ss} and {\AE}sop's {\OE}uvres
often na{\"\i}ve vis-\`a-vis the d{\ae}monic ph{\oe}nix's official r\^ole
in fluffy souffl\'es? }
\def\omitaccents{\let\moretext=\relax}
\def\sample{\table\text}
\def\UsualTest#1{{\def\fontname{#1 }\startfont\sample\names}}
\english
\def\LaTeX{L\kern-.36em%
        {\setbox\z@\hbox{T}%
         \vbox to\ht\z@{\hbox{$\rm\scriptstyle A$}%
                        \vss}%
        }%
        \kern-.15em%
        \TeX}
\catcode`\@=12

% RANDOM.TEX       v.1   (Donald Arseneau)
% Generating "random" numbers in TeX. 
%
% Random integers are generated in the range 1 to 2147483646 by the
% macro \nextrandom.  The result is returned in the counter \randomi.
% Do not change \randomi except, perhaps, to initialize it at some
% random value.  If you do not initialize it, it will be initialized
% using the time and date.  (This is a sparse initialization, giving
% fewer than a million different starting values, but you should use
% other sources of numbers if they are available--just remember that
% most of the numbers available to TeX are not at all random.)
%
% The \nextrandom command is not very useful by itself, unless you
% have exactly 2147483646 things to choose from.  Much more useful
% is the \setrannum command which sets a given counter to a random
% value within a specified range.  There are three parameters:
% \setrannum {<counter>} {<minimum>} {<maximum>}.  For example, to
% simulate a die-roll: \setrannum{\die}{1}{6} \ifcase\die... .
%
% If you need random numbers that are not integers, you will have to
% use dimen registers and \setrandimen.  For example, to set a random
% page width: \setrandimen \hsize{3in}{6.5in}.  The "\pointless" macro
% will remove the "pt" that TeX gives so you can use the dimensions
% as pure `real' numbers.  In that case, specify the range in pt units.
% For example,
%   \setrandimen\answer{2.71828pt}{3.14159pt}
%   The answer is \pointless\answer.
%
% The random number generator is the one by Lewis, Goodman, and Miller
% (1969) and used as "ran0" in "Numerical Recipies" using Schrage's
% method for avoiding overflows.  The multiplier is 16807 (7^5), the
% added constant is 0, and the modulus is 2147483647 (2^{31}-1).  The
% range of integers generated is 1 - 2147483646.  A smaller range would
% reduce the complexity of the macros a bit, but not much--most of the
% code deals with initialization and type-conversion.  On the other hand,
% the large range may be wasted due to the sparse seed initialization.

\newcount\randomi % the random number seed (while executing)
\global\randomi\catcode`\@  % scratch variable during definitions
\catcode`\@=11

\def\nextrandom{\begingroup
 \ifnum\randomi<\@ne % then initialize with time
    \global\randomi\time
    \global\multiply\randomi388 \global\advance\randomi\year
    \global\multiply\randomi31 \global\advance\randomi\day
    \global\multiply\randomi97 \global\advance\randomi\month
    \message{Randomizer (from random.tex) initialized to \the\randomi.}%
    \nextrandom \nextrandom \nextrandom
 \fi
 \count@ii\randomi
 \divide\count@ii 127773 % modulus = multiplier * 127773 + 2836
 \count@\count@ii
 \multiply\count@ii 127773
 \global\advance\randomi-\count@ii % random mod 127773
 \global\multiply\randomi 16807
 \multiply\count@ 2836
 \global\advance\randomi-\count@
 \ifnum\randomi<\z@ \global\advance\randomi 2147483647\relax\fi
 \endgroup
}

\countdef\count@ii=2 % use only in boxes!
\ifx\@tempcnta\undefined \csname newcount\endcsname \@tempcnta \fi
\ifx\@tempcntb\undefined \csname newcount\endcsname \@tempcntb \fi

\def\setrannum#1#2#3{% count register, minimum, maximum
 \@tempcnta#3\advance\@tempcnta-#2\advance\@tempcnta\@ne
 \@tempcntb 2147483645 %  =  m - 2  =  2^{31} - 3
 \divide\@tempcntb\@tempcnta
 \getr@nval
 \advance\ranval#2\relax
 #1\ranval
}

\def\setrandim#1#2#3{% dimen register, minimum length, maximum length
 \dimen@#2\dimen@ii#3\relax
 \setrannum\ranval\dimen@\dimen@ii
 #1\ranval sp\relax
}

\def\getr@nval{% The values in \@tempcnta and \@tempcntb are parameters
 \nextrandom
 \ranval\randomi \advance\ranval\m@ne \divide\ranval\@tempcntb
 \ifnum\ranval<\@tempcnta\else \expandafter\getr@nval \fi
}

\def\pointless{\expandafter\PoinTless\the}
{\catcode`p=12 \catcode`t=12 
\gdef\PoinTless#1pt{#1}}

\catcode`\@=\randomi
\global\randomi=0
\newcount\ranval

\endinput
